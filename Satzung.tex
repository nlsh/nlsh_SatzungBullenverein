% Satzung des Kleingartenverein "Bullenwiese" e.V.
 
\documentclass{scrartcl}

\usepackage[utf8]{inputenc}     % ein Muss für Deutsche Texte
\usepackage[T1]{fontenc}        % ein Muss für Deutsche Texte
\usepackage[ngerman]{babel}     % ein Muss für Deutsche Texte
\usepackage{lmodern}            % ein Muss für Deutsche Texte

% Zusätzliche Pakete
\usepackage{ulem}               % durchstreichen von Text
\usepackage{fancyhdr}           % Kopf- und Fußzeilen bearbeiten
\usepackage[juratotoc = true,   % Paragraphen (\Clause) ins Inhaltsverzeichnis
			contract	        % Vertragsmodus sofort einschalten
]{scrjura}  				    % Jura- Packet laden und aktivieren

% einzelne Sätze mit vorangesteltem "'S" Nummerieren 
\useshorthands{'}
\defineshorthand{'S}{\Sentence\ignorespaces}
\defineshorthand{'.}{. \Sentence\ignorespaces}

% Fancy Style für Kopf- und Fußzeile
\pagestyle{fancy}
\fancyhf{}
\rhead{Seite \thepage}
\lhead{Satzung Kleingartenverein „Bullenwiese“ Prenzlau e.V.}
\rfoot{Seite \thepage}


% Beginn
 
\title{\Huge Satzung}
\subtitle{\LARGE Kleingartenverein „Bullenwiese“ Prenzlau e.V.}
\author{Mitgliederversammlung}
\date{15. Oktober 2016}
 
\begin{document}

\maketitle
\tableofcontents

\newpage

\Clause{title = {Name, Sitz und Verbandszugehörigkeit}, number = 1}

	\begin{itemize}
		\item[1.] Der Verein führt den Namen Kleingartenverein „Bullenwiese“ Prenzlau
		\item[2.] Er hat seinen Sitz in Prenzlau und muss im Vereinsregister
				  eingetragen sein; er hat dann den Zusatz „e.V.“
		\item[3.] Der Verein muss Mitglied des zuständigen Kreisverbandes
		          „Gartenfreunde Prenzlau e.V.“ sein.
		\item[4.] Der Verein ist Rechtsnachfolger der 
				  „Kleingartensparte IX Bullenwiese“.
	\end{itemize}

		
\Clause{title = {Der Zweck des Vereins}, number = 2}

\begin{itemize}
	\item[1.] Der Kleingartenverein verfolgt ausschließlich und unmittelbar
			  gemeinnützige Zwecke im Sinne des Abschnitts „steuerbegünstigte Zwecke
			  der Abgabenordnung“ \\
			  Zweck des Vereines ist die Förderung des Kleingartenwesens und die
			  Förderung des Umwelt- und Landschaftsschutzes.
	\item[2.] Der Zweck wird verwirklicht durch	
		\begin{itemize}
			\item[1.] die Ausgestaltung der Kleingartenanlage als ein Bestandteil
					  des der Allgemeinheit zugänglichen, öffentlichen Grunds.
			\item[2.] die Zurverfügungstellung von Einzelgärten zur kleingärtnerischen
					  Betätigung,
			\item[3.] die Erziehung der Jugend zur Naturverbundenheit,
			\item[4.] die Eingliederung von Mitbürgern, um deren gesellschaftliche
				      Ausgrenzung zu vermeiden,
			\item[5.] die Zusammenfassung der Mitglieder in der Kleingartenanlage unter
					  Ausschluss jeglicher parteipolitischer oder konfessioneller
					  Ziele.
		\end{itemize}
\end{itemize}


\Clause{title = {Die Aufgaben des Vereins}, number = 3}

\begin{itemize}
	\item[1.] Darüber hinaus hat der Verein folgende Aufgaben:
		\begin{itemize}
		 	\item[1.] die Vergabe von Einzelparzellen an seine Mitglieder,
		 	\item[2.] die fachliche Beratung der Mitglieder,
		 \end{itemize}
	\item[2.] Der Kleingärtnerverein ist selbstlos tätig; er verfolgt nicht in erster
	          Linie eigenwirtschaftliche Zwecke.\\
              Mittel des Vereins dürfen nur für die satzungsgemäßen Zwecke verwendet
              werden.\\
              Mitglieder erhalten keine Zuwendungen aus Mitteln des Vereins.
	\item[3.] Es darf keine Person durch Ausgaben, die dem Zweck der Körperschaft fremd
			  sind, oder durch unverhältnismäßig hohe Vergütungen begünstigt werden.
\end{itemize}


\Clause{title = {Der Erwerb und die Beendigung der Vereinsmitgliedschaft}, number = 4} 

\begin{itemize}
	\item[1.] Aufnahme
		\begin{itemize}
		\item[1.] Mitglieder des Vereins können volljährige, am Kleingartenwesen
		          interessierte Personen werden.
		\item[2.] Die Aufnahme in den Verein ist schriftlich zu dokumentieren und wird
				  nach Zahlung vereinbarter Beiträge und mit Aushändigung einer
				  schriftlichen Aufnahmeerklärung wirksam. Die Ablehnung des
				  Aufnahmeantrags ist nicht anfechtbar. Ein Aufnahmeanspruch besteht
				  nicht.
		\item[3.] Mit der Aufnahme in die Vereinsgemeinschaft stehen den Mitgliedern
		          alle allgemeinen Mitgliedsrechte zu.
		\item[4.] Die Mitgliedschaft ist Voraussetzung für die Begründung eines
				  Kleingartenpachtverhältnisses mit dem Verein.
		\item[5.] Der Verein kann Ehrenmitglieder ernennen.
		\end{itemize}
	\item[2.] Beendigung
		\begin{itemize}
		 	\item[1.] Die Mitgliedschaft erlischt durch Tod, Austritt, Ausschluss oder
		 			  nach Vereinbarung.
		 	\item[2.] Der Austritt erfolgt durch schriftliche Erklärung des Mitgliedes.
		 	\item[3.] Ein Mitglied kann aus dem Verein ausgeschlossen werden, wenn ihm
		 			  gemäß §§ 8 oder 9 Abs. 1 Ziffer 1 Bundeskleingartengesetz der
		 			  Kleingarten gekündigt worden ist.

		\begin{quote}
		Diese lauten derzeit:
		
				§ 8: Kündigung ohne Einhaltung einer Kündigungsfrist
					\begin{itemize}
						\item[] Der Verpächter kann den Kleingartenpachtvertrag ohne
						    	Einhaltung einer Kündigungsfrist kündigen, wenn
							\begin{itemize}
						 		\item[1.] der Pächter mit der Entrichtung des
						 		          Pachtzinses für mindestens ein Vierteljahr in
						 		          Verzug ist und nicht innerhalb von zwei
						 		          Monaten nach schriftlicher Mahnung die fällige
						 				  Pachtzinsforderung erfüllt oder
						 		\item[2.] der Pächter oder von ihm auf dem
						 				  Kleingartengrundstück geduldete Personen so
						 				  schwerwiegende Pflichtverletzungen begehen,
						 				  insbesondere den Frieden in der
						 				  Kleingärtnergemeinschaft so nachhaltig stören,
						 				  dass dem Verpächter die Fortsetzung des
						 				  Vertragsverhältnisses nicht zugemutet werden
						 				  kann.
							\end{itemize} 
						\end{itemize}

				§ 9: Ordentliche Kündigung
					\begin{itemize}
						\item[] Der Verpächter kann den Kleingartenpachtvertrag
								kündigen, wenn
							\begin{itemize}
								\item[1.] der Pächter ungeachtet einer schriftlichen
										  Abmahnung des Verpächters eine nicht
										  kleingärtnerische Nutzung fortsetzt oder
										  andere Verpflichtungen, die die Nutzung des
										  Kleingartens betreffen, nicht unerheblich
										  verletzt, insbesondere die Laube zum dauernden
										  Wohnen benutzt, das Grundstück unbefugt einem
										  Dritten überlässt, erhebliche
										  Bewirtschaftungsmängel nicht innerhalb einer
										  angemessenen Frist abstellt oder geldliche
										  oder sonstige Gemeinschaftsleistungen für die
										  Kleingartenanlage verweigert.
								\item[2.] ...
							\end{itemize}
						\end{itemize}		
		\end{quote}
		\item[4.] Ein Mitglied kann auch aus dem Verein ausgeschlossen werden, wenn es
			\begin{enumerate}
				\item[1.] nach Fälligkeit und schriftlicher Mahnung mit der Zahlung von
						  Beiträgen und sonstigen Gemeinschaftsleistungen länger als
						  3 Monate im Rückstand ist,
				\item[2.] gegen die Bestimmungen dieser Satzung bzw. gegen die
						  Interessen des Vereins sowie gegen Beschlüsse und Anordnungen
						  der Vereinsorgane wiederholt verstößt,
				\item[3.] durch sein Verhalten die Gartengemeinschaft und das
						  Vereinsleben in erheblicher Weise stört.
			\end{enumerate}
		\item[5.] Über die Ausschließung entscheidet der Vorstand.
	\end{itemize}
\end{itemize}       


\Clause{title = {Die Rechte und Pflichten der Mitglieder}, number = 5}

\begin{itemize}
	\item[1.] Mit Begründung eines Kleingartenpachtverhältnisses erlangt das Mitglied
	          das Recht und die Pflicht zur kleingärtnerischen Nutzung; es ist kein
	 		  Sonderrecht i.S. des §~35~BGB. Dieses Recht kann das Mitglied für sich und
	 		  seine Familie ausüben. Es ist für ein nicht störendes Verhalten der
	 		  Familienmitglieder und seiner Besucher innerhalb der Gartengemeinschaft
	 		  verantwortlich.
	\item[2.] Nach Maßgabe dieser Satzung ist das Mitglied zur Betätigung innerhalb der
	 		  Gartengemeinschaft verpflichtet. Es hat Vereinsbeschlüsse zu beachten
	 		  sowie die Aufnahmegebühr, Beiträge und Umlagen termingerecht zu zahlen. Es
	 		  hat sich an der Gemeinschaftsarbeit zu beteiligen und als Abgeltung für
	 		  nicht geleistete Gemeinschaftsarbeit den hierfür vom Vorstand
	 		  festgesetzten Betrag zu entrichten.
	\item[3.] Zur Deckung außerplanmäßigen Finanzbedarfs über die gewöhnliche
			  Geschäftstätigkeit hinaus kann die Mitgliederversammlung die Erhebung von
			  Umlagen beschließen.
\end{itemize}

\pagebreak
\Clause{title = {Die Organe des Vereins}, number = 6}

\begin{itemize}
\item[1.] Die Organe des Vereins sind
	\begin{itemize}
	 	\item[1.] die Mitgliederversammlung
	 	\item[2.] der Vorstand
	 	\item[3.] für besondere Aufgaben können Ausschüsse gebildet werden.
	 \end{itemize} 
\end{itemize}


\Clause{title = {Die Mitgliederversammlung und ihre Aufgaben}, number = 7}

\begin{itemize}
	\item[1.] Die Mitgliederversammlung ist einzuberufen, wenn es das Vereinsinteresse
			  erfordert, mindestens jedoch alle 3 Jahre. Sie ist ferner zu berufen, wenn
			  ein Viertel der Mitglieder dieses schriftlich unter Angabe des Zwecks und
			  der Gründe verlangt.
	\item[2.] Mitgliederversammlungen sind durch den Vorsitzenden, im Verhinderungsfalle
			  durch seinen Stellvertreter, mit einer Frist von mindestens 14 Tagen
			  schriftlich mit Angabe von Ort, Zeit und Tagesordnung einzuberufen.
			  Aushang in der Gartenanlage genügt.
	\item[3.] Die Mitgliederversammlung beschließt in Vereinsangelegenheiten, soweit
			  hierfür nicht ein anderes Organ zuständig ist.
	
	Ihr obliegen vor allem:
		\begin{itemize}
			\item[1.]  Entgegennahme des Geschäftsberichtes, des Kassenberichtes, der
					   Berichte der Kassenprüfer,
			\item[2.]  Beschlussfassung über die Entlastung des Vorstandes,
			\item[3.]  Festsetzung der Aufnahmegebühr, des Jahresbeitrages, sonstiger
					   Beiträge und Umlagen sowie die Beschlussfassung über Rücklagen,
			\item[4.]  Wahl von Vorstandsmitgliedern,
			\item[5.]  Wahl eines Kassenprüfer und einem Ersatzmann, die unabhängig vom
				       Vorstand die Vereinskasse zu prüfen und hierüber zu berichten
				       haben,
			\item[6.]  Abberufung von Mitgliedern, die von der Mitgliederversammlung in
					   ein Amt gewählt worden sind,
			\item[7.]  Entscheidungen über Anträge und Beschwerden sowie über wichtige
					   Angelegenheiten, die ihr vom Vorstand unterbreitet werden,
			\item[8.]  Satzungsänderungen,
			\item[9.]  Auflösung des Vereins,
			\item[10.] Beschlussfassung über andere Angelegenheiten, soweit ihr diese
					   durch Satzungsbestimmungen zugewiesen sind.
		\end{itemize}
	\item[4.] Gültige Beschlüsse können nur zu Tagesordnungspunkten gefasst werden, die
			  den Mitgliedern mit der schriftlichen Einberufung der
			  Mitgliederversammlung bekannt gegeben wurden. Anträge zu den
			  Tagesordnungspunkten können schriftlich und mündlich jederzeit gestellt
			  werden.
	\item[5.] Ordnungsmäßig einberufene Mitgliederversammlungen sind – unabhängig von
			  der Zahl der erschienenen Mitglieder – beschlussfähig. Sie werden vom
			  Vorsitzenden, im Verhinderungsfall von seinem Stellvertreter, geleitet.
	\item[6.] Die Beschlussfassung erfolgt durch einfache Stimmenmehrheit der anwesenden
			  Mitglieder.\\
              Ungültige Stimmen bzw. Stimmenthaltungen werden nicht mitgezählt.
              Stimmengleichheit gilt als Ablehnung. Abgestimmt wird in der Regel durch
              Handzeichen, auf Antrag eines Drittels der anwesenden Mitglieder jedoch
              schriftlich durch Stimmzettel.\\
              Bei Angelegenheiten, die das Kleingartenpachtverhältnis betreffen, sind
              nur Mitglieder, die Pächter sind, stimmberechtigt. Bei solchen
              Abstimmungen zählt für jede Kleingartenparzelle nur eine Stimme. Bei einer
              Mehrzahl von Gartenpächtern kann die Stimme nur einheitlich abgegeben
              werden.
	\item[7.] Bei Wahlen gilt: Gewählt ist, wer in einer Abstimmung mehr als die Hälfte
			  der abgegebenen Stimmen erhält.\\
			  Ergibt sich keine einfache Stimmenmehrheit, so findet ein zweiter Wahlgang
			  statt, in dem gewählt ist, wer die meisten abgegebenen Stimmen erhält
			  (relative Mehrheit).\\
			  Bei Stimmengleichheit wird die Wahl wiederholt.\\
			  Bei erneuter Stimmengleichheit entscheidet das Los.
	\item[8.] Beschlüsse, durch welche die Satzung abgeändert wird, bedürfen der
	          Mehrheit von zwei Dritteln der anwesenden Mitglieder.
	\item[9.] Die Änderung des Zwecks sowie die Auflösung des Vereins können nur in
	          einer Mitgliederversammlung, welche hierzu besonders einzuberufen ist, mit
	          einer Mehrheit von drei Vierteln der erschienenen Mitglieder beschlossen
	          werden, wenn mindestens die Hälfte der Vereinsmitglieder hierbei anwesend
	          ist. Wird die erforderliche Anzahl nicht erreicht, wird in einer neu
	          einberufenen Mitgliederversammlung, unabhängig von der Anzahl der
	          erschienenen Mitglieder, mit Zweidrittelstimmenmehrheit beschlossen.
	\item[10.] Die Beschlüsse der Mitgliederversammlung sind binnen Monatsfrist zu
			   protokollieren und von dem Versammlungsleiter sowie dem Schriftführer zu
			   unterzeichnen. Jedes Mitglied ist berechtigt, das Protokoll einzusehen.
			   Es gilt als genehmigt, wenn innerhalb von 3 Monaten nach der
			   Mitgliederversammlung kein Widerspruch erfolgt. Kann ein Widerspruch
			   nicht ausgeräumt werden, entscheidet die Mitgliederversammlung hierüber.
\end{itemize}


\Clause{title = {Der Vorstand des Vereins und seine Zusammensetzung}, number = 8}

\begin{itemize}
	\item[1.] Der Verein wird von dem Vorstand geleitet.
	\item[2.] Dem Vorstand gehören an:
		\begin{enumerate}
			\item[1.] der Vorsitzende
			\item[2.] der Stellvertreter
			\item[3.] der Schriftführer
			\item[4.] der Kassierer
			\item[5.] der Fachberater		 
		\end{enumerate}
		Vorstandsmitglieder müssen Vereinsmitglieder sein; die Vereinigung mehrerer
		Vorstandsämter in einer Person ist unzulässig.
	\item[3.] Der Vorstand wird auf die Dauer von 3 Jahren gewählt.
			  Wiederwahl ist zulässig. Ein Vorstandsmitglied bleibt jedoch bis zur Wahl
			  eines Nachfolgers auf der nächsten Mitgliederversammlung im Amt.
	\item[4.] Scheidet ein Vorstandsmitglied vorzeitig aus, so ist in der nächsten
			  Mitgliederversammlung für die Restamtszeit eine Neuwahl vorzunehmen.
	\item[5.] Die Vorstandsmitglieder haften dem Verein nur für Vorsatz und grobe
			  Fahrlässigkeit.
	\item[6.] Vorstand im Sinne des § 26 BGB sind der Vorsitzende, sein Stellvertreter,
			  der Schriftführer,der Kassierer und der Fachberater.\\
			  Der Verein wird durch drei Vorstandsmitglieder im Sinne des § 26 BGB in
			  Gemeinschaft vertreten, von denen eines der Vorsitzende oder sein
			  Stellvertreter sein muss.
\end{itemize}


\Clause{title = {Das Verfahren in den Vorstandssitzungen und die Zuständigkeiten des Vorstandes}, number = 9}

\begin{itemize}
	\item[1.] Der Vorstand fasst seine Beschlüsse in Sitzungen, die von dem
	          Vorsitzenden,
		      im Verhinderungsfall von seinem Stellvertreter, einberufen und geleitet
		      werden.\\
			  Der Vorstand ist beschlussfähig, wenn mindestens 3 seiner Mitglieder
			  anwesend ist.\\
			  Er fasst seine Beschlüsse mit einfacher Stimmenmehrheit.\\
			  Bei Stimmengleichheit entscheidet die Stimme des Sitzungsleiters.
	\item[2.] Über die Sitzung des Vorstandes ist eine Niederschrift anzufertigen,
	          welche vom Sitzungsleiter und Protokollführer zu unterzeichnen ist.\\
			  Die Niederschrift ist in der nächsten Sitzung des Vorstandes auf
			  Verlangen bekannt zu geben.
	\item[3.] Sitzungen des Vorstandes sind bei Bedarf und mindestens 4 Mal im
			  Jahr einzuberufen.\\
	
	Dem Vorstand obliegen vor allem folgende Aufgaben:
		\begin{itemize}
			\item[1.] die Aufnahme neuer Vereinsmitglieder, 
			\item[2.] die Ausschließung von Vereinsmitgliedern, sofern sie nicht ein
					  Vorstandsamt oder ein sonstiges, ihnen von der 
					  Mitgliederversammlung übertragenes Amt bekleiden, 
			\item[3.] die Verpachtung des Kleingartens an Mitglieder,
			\item[4.] die Kündigung des Kleingartens gem. §§ 8 und 9 (1)
					  Bundeskleingartengesetz,
			\item[5.] die Schlichtung von Streitfällen aus dieser Satzung, sowie die
			          Erteilung von Verweisen und Verwarnungen,
			\item[6.] die Vorberatung von Angelegenheiten, die der Mitgliederversammlung
					  zur Beschlussfassung vorgelegt werden sollen,
			\item[7.] die Ernennung von Ehrenmitgliedern,
			\item[8.] die Festlegung der Gemeinschaftsarbeit einschließlich Vertretung
					  und finanzieller Abgeltung bei Säumnis,
			\item[9.] die Bestellung des Wertermittlers bzw. des
					  Wertermittlungsausschusses,
			\item[10.] die Behandlung von Einwänden des scheidenden Nutzungsberechtigten
					   gegen die Wertermittlung,
			\item[11.] die Erledigung besonderer Aufgaben, die ihm von der
					   Mitgliederversammlung übertragen werden,
			\item[12.] die Bestimmung der Gartenobleute und sonstiger Mitarbeiter,
			\item[13.] die Einrichtung und Besetzung von Ausschüssen zur Durchführung
			           von besonderen oder vorübergehenden Vereinsaufgaben,
			\item[14.] die Festlegung der Grundsätze der Gartenbewirtschaftung und -
					   gestaltung.
		\end{itemize}
	\item[4.] Der Vorstand veranlasst die zur Erfüllung des Vereinszwecks erforderlichen
			  Maßnahmen.\\
			  Er hält die Mitglieder dazu an, Ihre Pflichten in der Gartenanlage und im
			  Einzelgarten zu erfüllen. \\
			  Er bereitet die Sitzungen der Mitgliederversammlung vor.
	\item[5.] Der Schriftführer hat über jede Sitzung des Vorstandes und der
			  Mitgliederversammlung eine Niederschrift anzufertigen und darin die
			  Beschlüsse aufzuzeichnen. Die Niederschriften sind von ihm und dem
			  Sitzungs- oder Versammlungsleiter zu unterzeichnen.
	\item[6.] Der Kassierer verwaltet die Kasse des Vereins, führt ordnungsgemäß
			  Buch über alle Einnahmen und Ausgaben. Er weist Gegenstände und Geräte des
			  Vereins sowie dessen Vermögen in einem Verzeichnis nach und hat in
			  besonderen Fällen dem Vorstand einen mit Belegen versehenen Kassenbericht
			  vorzulegen.
	\item[7.] Die Vorstandsmitglieder haben den Kassenprüfern über die Geschäftsführung
			  Auskunft zu erteilen und ihnen in den Schriftverkehr sowie in Bücher,
			  Belege, Verzeichnisse und Bestände Einsicht zu gewähren. 
\end{itemize}


\Clause{title = {Die Aufwandsentschädigungen und Arbeitsverträge}, number = 10}

\begin{itemize}
	\item[1.] Alle Inhaber von Vereinsämtern sind grundsätzlich ehrenamtlich tätig.
			  Jedoch kann den Vorstandsmitgliedern und dem
			  Kassenprüfer der entstandene Aufwand entsprechend den
			  steuerrechtlichen Vorschriften erstattet werden.
	\item[2.] Falls jedoch die anfallenden Arbeiten das zumutbare Maß ehrenamtlicher
			  Tätigkeit übersteigen, können hauptamtliche Kräfte eingestellt werden.
			  Hier ist insbesondere auf die Angemessenheit der Vergütung ein besonderes
			  Augenmerk zu richten.
	\item[3.] Die bestellten Amtsträger des Vereins, insbesondere
			  Vorstandsmitglieder, können auf Beschluss der Mitgliederversammlung
			  angemessene Vergütungen für ihren Arbeits- oder Zeitaufwand
			  (Tätigkeitsvergütungen) erhalten.
\end{itemize}


\Clause{title = {Das Geschäftsjahr des Vereins}, number = 11}

Das Geschäftsjahr ist das Kalenderjahr.


\Clause{title = {Die Auflösung des Vereins}, number = 12}

\begin{itemize}
	\item[1.] Wird die Auflösung des Kleingärtnervereins oder die Änderung seines
	          Zweckes und der Aufgaben (§§ 2, 3) auf einer dafür einberufenen
			  Mitgliederversammlung in ordnungsmäßiger Weise beschlossen, so erfolgt die
			  Liquidation durch den Vorstand.
	\item[2.] Bei Auflösung oder Aufhebung des Vereins oder bei Wegfall seines
	          bisherigen Zwecks fällt das Vermögen des Vereins an den
			  Kreisverband ~„Gartenfreunde~Prenzlau~e.V.“,~ der es unmittelbar
			  und ausschließlich zur Förderung der unter § 2 der Satzung genannten
			  Zwecke (Förderung des Kleingartenwesens) zu verwenden hat. 
\end{itemize}

\pagebreak
\section*{Schlussbestimmungen}


\Clause{title = {Die Aufhebung der bisherigen Satzung}, number = 13}

Die Regelungen der bisherigen Satzung werden aufgehoben und durch diese ersetzt.


\Clause{title = {Das Recht des Vorstandes zur Satzungsänderung oder Ergänzung}, number = 14}

\begin{itemize}
	\item[1.] Der Vorstand ist berechtigt, Ergänzungen redaktioneller Art selbständig
			  vorzunehmen, auch soweit sie vom Registergericht gefordert werden.	
	\item[2.] Angenommen in der Mitgliederversammlung\\[0,5cm]
	          in Prenlau, am 15. Oktober 2016
	\item[3.] Eingetragen im Vereinsregister Amtsgericht:\\[0,5cm]
              Neuruppin, den 08. Dezember 2016
\end{itemize}



\end{document}
